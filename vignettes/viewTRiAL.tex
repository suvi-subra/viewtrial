\documentclass[]{article}
\usepackage{lmodern}
\usepackage{amssymb,amsmath}
\usepackage{ifxetex,ifluatex}
\usepackage{fixltx2e} % provides \textsubscript
\ifnum 0\ifxetex 1\fi\ifluatex 1\fi=0 % if pdftex
  \usepackage[T1]{fontenc}
  \usepackage[utf8]{inputenc}
\else % if luatex or xelatex
  \ifxetex
    \usepackage{mathspec}
  \else
    \usepackage{fontspec}
  \fi
  \defaultfontfeatures{Ligatures=TeX,Scale=MatchLowercase}
\fi
% use upquote if available, for straight quotes in verbatim environments
\IfFileExists{upquote.sty}{\usepackage{upquote}}{}
% use microtype if available
\IfFileExists{microtype.sty}{%
\usepackage{microtype}
\UseMicrotypeSet[protrusion]{basicmath} % disable protrusion for tt fonts
}{}
\usepackage[margin=1in]{geometry}
\usepackage{hyperref}
\hypersetup{unicode=true,
            pdftitle={viewTRiAL Package - A technical guide},
            pdfauthor={Suvitha Subramaniam \& Katharina Klatte},
            pdfborder={0 0 0},
            breaklinks=true}
\urlstyle{same}  % don't use monospace font for urls
\usepackage{color}
\usepackage{fancyvrb}
\newcommand{\VerbBar}{|}
\newcommand{\VERB}{\Verb[commandchars=\\\{\}]}
\DefineVerbatimEnvironment{Highlighting}{Verbatim}{commandchars=\\\{\}}
% Add ',fontsize=\small' for more characters per line
\usepackage{framed}
\definecolor{shadecolor}{RGB}{248,248,248}
\newenvironment{Shaded}{\begin{snugshade}}{\end{snugshade}}
\newcommand{\AlertTok}[1]{\textcolor[rgb]{0.94,0.16,0.16}{#1}}
\newcommand{\AnnotationTok}[1]{\textcolor[rgb]{0.56,0.35,0.01}{\textbf{\textit{#1}}}}
\newcommand{\AttributeTok}[1]{\textcolor[rgb]{0.77,0.63,0.00}{#1}}
\newcommand{\BaseNTok}[1]{\textcolor[rgb]{0.00,0.00,0.81}{#1}}
\newcommand{\BuiltInTok}[1]{#1}
\newcommand{\CharTok}[1]{\textcolor[rgb]{0.31,0.60,0.02}{#1}}
\newcommand{\CommentTok}[1]{\textcolor[rgb]{0.56,0.35,0.01}{\textit{#1}}}
\newcommand{\CommentVarTok}[1]{\textcolor[rgb]{0.56,0.35,0.01}{\textbf{\textit{#1}}}}
\newcommand{\ConstantTok}[1]{\textcolor[rgb]{0.00,0.00,0.00}{#1}}
\newcommand{\ControlFlowTok}[1]{\textcolor[rgb]{0.13,0.29,0.53}{\textbf{#1}}}
\newcommand{\DataTypeTok}[1]{\textcolor[rgb]{0.13,0.29,0.53}{#1}}
\newcommand{\DecValTok}[1]{\textcolor[rgb]{0.00,0.00,0.81}{#1}}
\newcommand{\DocumentationTok}[1]{\textcolor[rgb]{0.56,0.35,0.01}{\textbf{\textit{#1}}}}
\newcommand{\ErrorTok}[1]{\textcolor[rgb]{0.64,0.00,0.00}{\textbf{#1}}}
\newcommand{\ExtensionTok}[1]{#1}
\newcommand{\FloatTok}[1]{\textcolor[rgb]{0.00,0.00,0.81}{#1}}
\newcommand{\FunctionTok}[1]{\textcolor[rgb]{0.00,0.00,0.00}{#1}}
\newcommand{\ImportTok}[1]{#1}
\newcommand{\InformationTok}[1]{\textcolor[rgb]{0.56,0.35,0.01}{\textbf{\textit{#1}}}}
\newcommand{\KeywordTok}[1]{\textcolor[rgb]{0.13,0.29,0.53}{\textbf{#1}}}
\newcommand{\NormalTok}[1]{#1}
\newcommand{\OperatorTok}[1]{\textcolor[rgb]{0.81,0.36,0.00}{\textbf{#1}}}
\newcommand{\OtherTok}[1]{\textcolor[rgb]{0.56,0.35,0.01}{#1}}
\newcommand{\PreprocessorTok}[1]{\textcolor[rgb]{0.56,0.35,0.01}{\textit{#1}}}
\newcommand{\RegionMarkerTok}[1]{#1}
\newcommand{\SpecialCharTok}[1]{\textcolor[rgb]{0.00,0.00,0.00}{#1}}
\newcommand{\SpecialStringTok}[1]{\textcolor[rgb]{0.31,0.60,0.02}{#1}}
\newcommand{\StringTok}[1]{\textcolor[rgb]{0.31,0.60,0.02}{#1}}
\newcommand{\VariableTok}[1]{\textcolor[rgb]{0.00,0.00,0.00}{#1}}
\newcommand{\VerbatimStringTok}[1]{\textcolor[rgb]{0.31,0.60,0.02}{#1}}
\newcommand{\WarningTok}[1]{\textcolor[rgb]{0.56,0.35,0.01}{\textbf{\textit{#1}}}}
\usepackage{graphicx}
% grffile has become a legacy package: https://ctan.org/pkg/grffile
\IfFileExists{grffile.sty}{%
\usepackage{grffile}
}{}
\makeatletter
\def\maxwidth{\ifdim\Gin@nat@width>\linewidth\linewidth\else\Gin@nat@width\fi}
\def\maxheight{\ifdim\Gin@nat@height>\textheight\textheight\else\Gin@nat@height\fi}
\makeatother
% Scale images if necessary, so that they will not overflow the page
% margins by default, and it is still possible to overwrite the defaults
% using explicit options in \includegraphics[width, height, ...]{}
\setkeys{Gin}{width=\maxwidth,height=\maxheight,keepaspectratio}
\IfFileExists{parskip.sty}{%
\usepackage{parskip}
}{% else
\setlength{\parindent}{0pt}
\setlength{\parskip}{6pt plus 2pt minus 1pt}
}
\setlength{\emergencystretch}{3em}  % prevent overfull lines
\providecommand{\tightlist}{%
  \setlength{\itemsep}{0pt}\setlength{\parskip}{0pt}}
\setcounter{secnumdepth}{5}
% Redefines (sub)paragraphs to behave more like sections
\ifx\paragraph\undefined\else
\let\oldparagraph\paragraph
\renewcommand{\paragraph}[1]{\oldparagraph{#1}\mbox{}}
\fi
\ifx\subparagraph\undefined\else
\let\oldsubparagraph\subparagraph
\renewcommand{\subparagraph}[1]{\oldsubparagraph{#1}\mbox{}}
\fi

%%% Use protect on footnotes to avoid problems with footnotes in titles
\let\rmarkdownfootnote\footnote%
\def\footnote{\protect\rmarkdownfootnote}

%%% Change title format to be more compact
\usepackage{titling}

% Create subtitle command for use in maketitle
\providecommand{\subtitle}[1]{
  \posttitle{
    \begin{center}\large#1\end{center}
    }
}

\setlength{\droptitle}{-2em}

  \title{\textbf{viewTRiAL} Package - A technical guide}
    \pretitle{\vspace{\droptitle}\centering\huge}
  \posttitle{\par}
    \author{Suvitha Subramaniam \& Katharina Klatte}
    \preauthor{\centering\large\emph}
  \postauthor{\par}
      \predate{\centering\large\emph}
  \postdate{\par}
    \date{2020-09-28}


\begin{document}
\maketitle

{
\setcounter{tocdepth}{3}
\tableofcontents
}
\newpage

\hypertarget{about-viewtrial}{%
\section{\texorpdfstring{\textbf{About
viewTRiAL}}{About viewTRiAL}}\label{about-viewtrial}}

The viewTRiAL package is a R Shiny (package \{shiny\} version 1.5.0) web
application prototype designed to build and deploy Centralized
Monitoring R shiny web applications in clinical studies within the
Department of Clinical Research at University of Basel.

\hypertarget{install-viewtrial}{%
\section{\texorpdfstring{\textbf{Install
viewTRiAL}}{Install viewTRiAL}}\label{install-viewtrial}}

\hypertarget{installing-from-ctus-repository}{%
\subsection{Installing from CTU's
repository}\label{installing-from-ctus-repository}}

\begin{Shaded}
\begin{Highlighting}[]

\KeywordTok{setwd}\NormalTok{(}\StringTok{"~repos/proj/piCentralizedMonitoring/R"}\NormalTok{)}
\KeywordTok{system}\NormalTok{(}\StringTok{"R CMD INSTALL viewTRiAL"}\NormalTok{)}
\end{Highlighting}
\end{Shaded}

\hypertarget{installing-from-github}{%
\subsection{Installing from GitHub}\label{installing-from-github}}

\begin{Shaded}
\begin{Highlighting}[]

\KeywordTok{library}\NormalTok{(devtools)}
\NormalTok{devtools}\OperatorTok{::}\KeywordTok{install_github}\NormalTok{(}\StringTok{"suvi-subra/viewTRiAL"}\NormalTok{)}
\end{Highlighting}
\end{Shaded}

\hypertarget{running-viewtrial}{%
\subsection{Running viewTRiAL}\label{running-viewtrial}}

\begin{Shaded}
\begin{Highlighting}[]

\KeywordTok{library}\NormalTok{(viewTRiAL)}
\KeywordTok{run_app}\NormalTok{()}
\end{Highlighting}
\end{Shaded}

\hypertarget{lets-get-started}{%
\section{\texorpdfstring{\textbf{Let's get
started!}}{Let's get started!}}\label{lets-get-started}}

The prototype has been developed as Shiny modules to address namespacing
problems in Shiny User interface (UI) and server logic and to facilitate
working through huge amount of code. In order to develop a
production-grade Shiny App, the framework from the \{golem\} package
version 0.2.1 was used. The structure of a golem framework is the same
as that of a typical R package except that it allows for simplified
development and deployment of packaged R Shiny applications.

\hypertarget{load-prepare-and-save-your-data-data-raw}{%
\subsection{\texorpdfstring{\textbf{Load, prepare and save your data}
(data-raw/)}{Load, prepare and save your data (data-raw/)}}\label{load-prepare-and-save-your-data-data-raw}}

For maximum efficiency in running your app, each time you launch it, you
do not want it to be loaded more than once. Since we are packaging our R
Shiny app, we can take advantage of storing data as internal data,
making it unavailable to users. A script that loads, prepares and saves
the data as internal data is stored under the data-raw/ folder. This
script does not run automatically when the app is launched. Therefore, a
line to run the script can be incorporated in the app\_server.R file.

We use 4 dummy datasets in this prototype app. The structure of each
dataset is displayed.

\hypertarget{baseline-data}{%
\subsubsection{Baseline data}\label{baseline-data}}

This is baseline data containing one row for each patient. The pat\_id,
centre, withdrawal status, randomization date and informed consent dates
are included.

\begin{Shaded}
\begin{Highlighting}[]

\KeywordTok{str}\NormalTok{(dummy_baseline)}
\CommentTok{#> 'data.frame':    100 obs. of  5 variables:}
\CommentTok{#>  $ pat_id           : int  100 51 13 40 90 50 31 79 7 61 ...}
\CommentTok{#>  $ centre           : Factor w/ 5 levels "A","B","C","D",..: 4 3 5 3 2 5 2 2 3 1 ...}
\CommentTok{#>  $ withdrawn        : logi  FALSE FALSE FALSE FALSE FALSE FALSE ...}
\CommentTok{#>  $ date_random.date : Date, format: "2019-08-15" "2019-08-23" "2020-04-28" "2020-06-04" ...}
\CommentTok{#>  $ date_consent.date: Date, format: "2019-08-15" "2019-08-23" "2020-04-28" "2020-06-04" ...}
\end{Highlighting}
\end{Shaded}

\hypertarget{track-your-development-dev}{%
\subsection{\texorpdfstring{\textbf{Track your development}
(dev/)}{Track your development (dev/)}}\label{track-your-development-dev}}

The dev/ folder is used as a notebook to track steps of the development
process. There are 3 files:

\hypertarget{start.r}{%
\subsubsection{01\_start.R}\label{start.r}}

This file should be filled at the start. It contains functions to fill
the DESCRIPTION file, set options adn dependencies.

\hypertarget{dev.r}{%
\subsubsection{02\_dev.R}\label{dev.r}}

This file should be filled next to set the structure of core
functionalities such as creating modules and other R functions.

\hypertarget{deploy.r}{%
\subsubsection{03\_deploy.R}\label{deploy.r}}

Finally, this file contains functions that aid in deploying the app via
Rstudio or Docker. This file contains functions to create either an
app.R file or Docker file to deploy the app depending on the environment
you choose. This is a necessary step as the R Shiny web application is
packaged.

\hypertarget{start-developing-r}{%
\subsection{\texorpdfstring{\textbf{Start developing}
(R/)}{Start developing (R/)}}\label{start-developing-r}}

Core functions of the app are stored under the R/ folder.

\hypertarget{app_ui.r}{%
\subsubsection{app\_ui.R}\label{app_ui.r}}

This function displays the input from its counterpart app\_server().
Below is an example of how a menu item in the sidebar of a dashboard is
created and its corresponding module called to display the content on
the dashboard body.

\begin{Shaded}
\begin{Highlighting}[]

\NormalTok{app_ui <-}\StringTok{ }\ControlFlowTok{function}\NormalTok{(request) \{}
  
  \KeywordTok{fluidPage}\NormalTok{(}
    \KeywordTok{dashboardPage}\NormalTok{(}\DataTypeTok{skin =} \StringTok{"purple"}\NormalTok{,}
    
      \KeywordTok{dashboardHeader}\NormalTok{(}\DataTypeTok{title =} \StringTok{"Study name"}\NormalTok{, }\DataTypeTok{titleWidth =} \DecValTok{300}\NormalTok{),}
      \KeywordTok{dashboardSidebar}\NormalTok{(}
        \KeywordTok{sidebarMenu}\NormalTok{(}
          \CommentTok{## 1. Create a new menuitem here. List tabName in get_modules.R.}
          \KeywordTok{menuItem}\NormalTok{(}\StringTok{"Overview"}\NormalTok{, }\DataTypeTok{tabName =}\NormalTok{ mod}\OperatorTok{$}\NormalTok{overview, }\DataTypeTok{icon =} \KeywordTok{icon}\NormalTok{(}\StringTok{"chart-pie"}\NormalTok{))}
\NormalTok{        )}
\NormalTok{      ),}
      \KeywordTok{dashboardBody}\NormalTok{(}
        \KeywordTok{tabItems}\NormalTok{(}
          \CommentTok{## 2. Call the module corresponding to the menu item here}
          \KeywordTok{mod_overview_ui}\NormalTok{(}\DataTypeTok{id =}\NormalTok{ mod}\OperatorTok{$}\NormalTok{overview, }\DataTypeTok{label =}\NormalTok{ mod}\OperatorTok{$}\NormalTok{overview)}
\NormalTok{        )}
\NormalTok{      )}
      
\NormalTok{    )}
\NormalTok{  )}
\NormalTok{\}}
\end{Highlighting}
\end{Shaded}

There are 4 items to adapt in this file:

\begin{enumerate}
\def\labelenumi{\arabic{enumi}.}
\tightlist
\item
  dashboardHeader(title): Change the study name
\item
  dateRangeInput(inputId): Replace the name of the date filter depending
  on the date used to filter the data. (eg. Enrolment date). This filter
  may be commented if not needed.
\item
  dateRangeInput(label): Replace the date ranges for filtering.
\item
  selectInput(choices, selected): Replace the vector of centers and
  default center to be selected. This filter may be commented if not
  needed.
\end{enumerate}

\begin{Shaded}
\begin{Highlighting}[]

\NormalTok{app_ui <-}\StringTok{ }\ControlFlowTok{function}\NormalTok{(request) \{}
  
    \CommentTok{## Get module label names}
\NormalTok{    mod <-}\StringTok{ }\KeywordTok{get_modules}\NormalTok{()}
  
    \CommentTok{## Leave this function for adding external resources}
    \KeywordTok{golem_add_external_resources}\NormalTok{()}
    \CommentTok{## List the first level UI elements here }
    \KeywordTok{fluidPage}\NormalTok{(}
      \KeywordTok{dashboardPage}\NormalTok{(}\DataTypeTok{skin =} \StringTok{"purple"}\NormalTok{,}
      \CommentTok{## Header}
      \CommentTok{## }\AlertTok{TODO}\CommentTok{ (1): Change study name}
      \KeywordTok{dashboardHeader}\NormalTok{(}\DataTypeTok{title =} \StringTok{"Study name"}\NormalTok{, }\DataTypeTok{titleWidth =} \DecValTok{300}\NormalTok{),}
      \CommentTok{## Sidebar}
      \KeywordTok{dashboardSidebar}\NormalTok{(}\DataTypeTok{width =} \DecValTok{300}\NormalTok{,}
        \KeywordTok{sidebarMenu}\NormalTok{(}
            \CommentTok{## TAB 1: Overview}
            \KeywordTok{menuItem}\NormalTok{(}\StringTok{"Overview"}\NormalTok{, }\DataTypeTok{tabName =}\NormalTok{ mod}\OperatorTok{$}\NormalTok{overview, }\DataTypeTok{icon =} \KeywordTok{icon}\NormalTok{(}\StringTok{"chart-pie"}\NormalTok{)),}
            \CommentTok{## TAB 2: Performance measures}
            \KeywordTok{menuItem}\NormalTok{(}\StringTok{"Performance measures"}\NormalTok{, }\DataTypeTok{icon =} \KeywordTok{icon}\NormalTok{(}\StringTok{"chart-pie"}\NormalTok{),}
              \CommentTok{## TAB 2.1: Recruitment and retention}
              \KeywordTok{menuSubItem}\NormalTok{(}\StringTok{"Recruitment and Retention"}\NormalTok{, }\DataTypeTok{tabName =}\NormalTok{ mod}\OperatorTok{$}\NormalTok{recruitment),}
              \CommentTok{## TAB 2.2: Informed consent and eligibility}
              \KeywordTok{menuSubItem}\NormalTok{(}\StringTok{"Informed consent and Eligibility"}\NormalTok{, }\DataTypeTok{tabName =}\NormalTok{ mod}\OperatorTok{$}\NormalTok{consent),}
              \CommentTok{## TAB 2.3: Data quality}
              \KeywordTok{menuSubItem}\NormalTok{(}\StringTok{"Data Quality"}\NormalTok{, }\DataTypeTok{tabName =}\NormalTok{ mod}\OperatorTok{$}\NormalTok{quality)),}
            \CommentTok{## TAB 3: Study management}
            \KeywordTok{menuItem}\NormalTok{(}\StringTok{"Study Management"}\NormalTok{, }\DataTypeTok{icon =} \KeywordTok{icon}\NormalTok{(}\StringTok{"chart-pie"}\NormalTok{),}
              \CommentTok{## TAB 3.1: Visits}
              \KeywordTok{menuSubItem}\NormalTok{(}\StringTok{"Visits"}\NormalTok{, }\DataTypeTok{tabName =}\NormalTok{ mod}\OperatorTok{$}\NormalTok{visits),}
              \CommentTok{## TAB 3.2: Biosampling}
              \KeywordTok{menuSubItem}\NormalTok{(}\StringTok{"Biosampling/Imaging data"}\NormalTok{, }\DataTypeTok{tabName =}\NormalTok{ mod}\OperatorTok{$}\NormalTok{lab),}
              \CommentTok{## TAB 3.3: Safety}
              \KeywordTok{menuSubItem}\NormalTok{(}\StringTok{"Safety management"}\NormalTok{, }\DataTypeTok{tabName =}\NormalTok{ mod}\OperatorTok{$}\NormalTok{safety),}
              \CommentTok{## TAB 3.4: Staff management}
              \KeywordTok{menuSubItem}\NormalTok{(}\StringTok{"Staff management"}\NormalTok{, }\DataTypeTok{tabName =}\NormalTok{ mod}\OperatorTok{$}\NormalTok{staff)),}
            \CommentTok{## FILTER: Date range }
            \CommentTok{## }\AlertTok{TODO}\CommentTok{ (2): Replace 2nd argument "Randomization date:" }
            \KeywordTok{dateRangeInput}\NormalTok{(}\DataTypeTok{inputId =}  \StringTok{"period"}\NormalTok{, }\DataTypeTok{label =} \StringTok{"Randomization date:"}\NormalTok{,}
              \CommentTok{## }\AlertTok{TODO}\CommentTok{ (3): Replace date ranges }
              \DataTypeTok{start =} \KeywordTok{as.POSIXct}\NormalTok{(}\StringTok{"2019-06-01"}\NormalTok{),}
              \DataTypeTok{end   =} \KeywordTok{as.POSIXct}\NormalTok{(}\KeywordTok{today}\NormalTok{())),}
            \CommentTok{## FILTER: Center}
            \CommentTok{## }\AlertTok{TODO}\CommentTok{ (4): Replace 3rd argument "choices" with a vector of centers }
            \KeywordTok{selectInput}\NormalTok{(}\StringTok{"center"}\NormalTok{, }\StringTok{"Center"}\NormalTok{, }
                          \DataTypeTok{choices =} \KeywordTok{c}\NormalTok{(}\StringTok{"All"}\NormalTok{, }\KeywordTok{sort}\NormalTok{(}\KeywordTok{levels}\NormalTok{(dummy_baseline}\OperatorTok{$}\NormalTok{centre))), }
                          \DataTypeTok{selected =} \StringTok{"All"}\NormalTok{),}
              \DataTypeTok{width =} \StringTok{"350"}\NormalTok{)),}
        \CommentTok{## Body}
        \KeywordTok{dashboardBody}\NormalTok{(}
          \KeywordTok{tabItems}\NormalTok{(}
            \CommentTok{## TAB 1: Overview}
            \KeywordTok{mod_overview_ui}\NormalTok{(mod}\OperatorTok{$}\NormalTok{overview, }\DataTypeTok{label =}\NormalTok{ mod}\OperatorTok{$}\NormalTok{overview),}
            \CommentTok{## TAB 2.1: Recruitment and retention}
            \KeywordTok{mod_recruit_ui}\NormalTok{(mod}\OperatorTok{$}\NormalTok{recruitment, }\DataTypeTok{label =}\NormalTok{ mod}\OperatorTok{$}\NormalTok{recruitment),}
            \CommentTok{## TAB 2.2: Informed consent and eligibility}
            \KeywordTok{mod_consent_ui}\NormalTok{(mod}\OperatorTok{$}\NormalTok{consent, }\DataTypeTok{label =}\NormalTok{ mod}\OperatorTok{$}\NormalTok{consent)}
\NormalTok{                   )}
\NormalTok{        ) }\CommentTok{## dashboardBody}
\NormalTok{      ) }\CommentTok{## dashboardPage}
\NormalTok{    ) }\CommentTok{## fluidPage}

\NormalTok{\}}
\end{Highlighting}
\end{Shaded}

\hypertarget{app_server.r}{%
\subsubsection{app\_server.R}\label{app_server.r}}

This file contains the server logic. In this file, reactive dataframes
are created and a callModule() corresponding to the module UI created in
app\_ui.R is called. Starting in Shiny 1.5.0, it is recommended to use
moduleServer() instead of callModule() as it can be tested with
testServer(). However, in our prototype since we have additional
parameters to pass to each module function, we will stick to using
callModule() as moduleServer() will not be able to handle that.

There are 4 items to adapt in this file:

\begin{enumerate}
\def\labelenumi{\arabic{enumi}.}
\tightlist
\item
  The input dataframes in both filter functions and callModule(data1)
\item
  The variable names of centre and filter date
\end{enumerate}

\begin{Shaded}
\begin{Highlighting}[]

\CommentTok{#' The application server-side}
\CommentTok{#' }
\CommentTok{#' @param input,output,session Internal parameters for \{shiny\}. }
\CommentTok{#'     DO NOT REMOVE.}
\CommentTok{#' @import shiny}
\CommentTok{#' @noRd}
\NormalTok{app_server <-}\StringTok{ }\ControlFlowTok{function}\NormalTok{(input, output, session) \{}
  
  \CommentTok{## Get module label names}
\NormalTok{  mod <-}\StringTok{ }\KeywordTok{get_modules}\NormalTok{()}
  
  \CommentTok{## Create reactive dataframes}
\NormalTok{  bl_period <-}\StringTok{ }\KeywordTok{reactive}\NormalTok{(}
    
    \ControlFlowTok{if}\NormalTok{(input}\OperatorTok{$}\NormalTok{center }\OperatorTok{!=}\StringTok{ "All"}\NormalTok{)\{}
      \CommentTok{## }\AlertTok{TODO}\CommentTok{ (1): Replace input dataframe  }
      \CommentTok{## }\AlertTok{TODO}\CommentTok{ (2): Change variable names "centre" and "date.random.date"}
      \KeywordTok{filter}\NormalTok{(}\DataTypeTok{df =}\NormalTok{ dummy_baseline, }
\NormalTok{             centre }\OperatorTok{==}\StringTok{ }\NormalTok{input}\OperatorTok{$}\NormalTok{center }\OperatorTok{&}\StringTok{ }
\StringTok{               }\NormalTok{date_random.date }\OperatorTok{>=}\StringTok{ }\NormalTok{input}\OperatorTok{$}\NormalTok{period[}\DecValTok{1}\NormalTok{] }\OperatorTok{&}\StringTok{ }
\StringTok{               }\NormalTok{date_random.date }\OperatorTok{<=}\StringTok{ }\NormalTok{input}\OperatorTok{$}\NormalTok{period[}\DecValTok{2}\NormalTok{])}
      
\NormalTok{    \} }\ControlFlowTok{else}\NormalTok{\{}
      \CommentTok{## }\AlertTok{TODO}\CommentTok{ (3): Replace input dataframe }
      \CommentTok{## }\AlertTok{TODO}\CommentTok{ (4): Change variable names "centre" and "date.random.date"}
      \KeywordTok{filter}\NormalTok{(}\DataTypeTok{df =}\NormalTok{ dummy_baseline, }
\NormalTok{             date_random.date }\OperatorTok{>=}\StringTok{ }\NormalTok{input}\OperatorTok{$}\NormalTok{period[}\DecValTok{1}\NormalTok{] }\OperatorTok{&}\StringTok{ }
\StringTok{               }\NormalTok{date_random.date }\OperatorTok{<=}\StringTok{ }\NormalTok{input}\OperatorTok{$}\NormalTok{period[}\DecValTok{2}\NormalTok{])}
\NormalTok{    \})}
  
  \CommentTok{# List the first level callModules here}
  \CommentTok{## Overview tab}
  \CommentTok{## }\AlertTok{TODO}\CommentTok{ (5): Replace input dataframe in argument data1}
  \KeywordTok{callModule}\NormalTok{(mod_overview_server, mod}\OperatorTok{$}\NormalTok{overview, }\DataTypeTok{data1 =}\NormalTok{ bl_period)}
  \KeywordTok{callModule}\NormalTok{(mod_recruit_server, mod}\OperatorTok{$}\NormalTok{recruitment, }\DataTypeTok{data1 =}\NormalTok{ bl_period)}
  \KeywordTok{callModule}\NormalTok{(mod_consent_server, mod}\OperatorTok{$}\NormalTok{consent, }\DataTypeTok{data1 =}\NormalTok{ bl_period)}
  
\NormalTok{\}}
\end{Highlighting}
\end{Shaded}

\hypertarget{app_config.r}{%
\subsubsection{app\_config.R}\label{app_config.r}}

This file contains a get\_golem\_config() that retrieves app config. No
changes to make here.

\hypertarget{get_modules.r}{%
\subsubsection{get\_modules.R}\label{get_modules.r}}

This file contains a list of module label names. No changes to make here
unless you add a new module.

\hypertarget{mod_overview.r}{%
\subsubsection{mod\_overview.R}\label{mod_overview.r}}

This file contains the ui and server functions for the ``Overview'' tab.
The ``Overview'' tab contains the following:

\begin{itemize}
\tightlist
\item
  value boxes displaying the number of patients, randomized, currently
  active and withdrawn
\item
  line plot displaying recruitment over time by centers
\end{itemize}

Here there are no changes required unless one wants to add value boxes
for additional figures or plots.

\begin{Shaded}
\begin{Highlighting}[]

\CommentTok{#' overview UI Function}
\CommentTok{#'}
\CommentTok{#' @description A shiny Module.}
\CommentTok{#'}
\CommentTok{#' @param id,input,output,session Internal parameters for \{shiny\}.}
\CommentTok{#'}
\CommentTok{#' @noRd }
\CommentTok{#'}
\CommentTok{#' @importFrom shiny NS tagList }
\NormalTok{mod_overview_ui <-}\StringTok{ }\ControlFlowTok{function}\NormalTok{(id, label)\{}
\NormalTok{  ns <-}\StringTok{ }\KeywordTok{NS}\NormalTok{(id)}
  
    \KeywordTok{tabItem}\NormalTok{(}\DataTypeTok{tabName =}\NormalTok{ label,}
            \KeywordTok{fluidRow}\NormalTok{(}
              \CommentTok{## No.of participants randomized}
              \KeywordTok{valueBoxOutput}\NormalTok{(}\KeywordTok{ns}\NormalTok{(}\StringTok{"randomized"}\NormalTok{), }\DataTypeTok{width =} \DecValTok{12}\NormalTok{)}
\NormalTok{            ),}
            
            \KeywordTok{fluidRow}\NormalTok{(}
              \KeywordTok{valueBoxOutput}\NormalTok{(}\KeywordTok{ns}\NormalTok{(}\StringTok{"active"}\NormalTok{), }\DataTypeTok{width =} \DecValTok{6}\NormalTok{),}
              \KeywordTok{valueBoxOutput}\NormalTok{(}\KeywordTok{ns}\NormalTok{(}\StringTok{"withdrew"}\NormalTok{), }\DataTypeTok{width =} \DecValTok{6}\NormalTok{)}
\NormalTok{            ),}
            
            \KeywordTok{fluidRow}\NormalTok{(}
              \CommentTok{## Plot displaying no.of participants randomized across acute centers}
              \KeywordTok{box}\NormalTok{(}
                \DataTypeTok{width =} \DecValTok{6}\NormalTok{,}
                \DataTypeTok{height =} \StringTok{"600"}\NormalTok{,}
                \DataTypeTok{title =} \StringTok{"Recruitment by centers"}\NormalTok{,}
                \DataTypeTok{status =} \StringTok{"primary"}\NormalTok{,}
                \KeywordTok{plotlyOutput}\NormalTok{(}\KeywordTok{ns}\NormalTok{(}\StringTok{'recruitplot'}\NormalTok{), }\DataTypeTok{height =} \StringTok{"500"}\NormalTok{),}
                \DataTypeTok{solidHeader =} \OtherTok{TRUE}\NormalTok{,}
                \DataTypeTok{collapsible =} \OtherTok{FALSE}\NormalTok{),}
              
\NormalTok{            )}
\NormalTok{    )}

\NormalTok{\}}
    
\CommentTok{#' overview Server Function}
\CommentTok{#'}
\CommentTok{#' @noRd }
\NormalTok{mod_overview_server <-}\StringTok{ }\ControlFlowTok{function}\NormalTok{(input, output, session, baseline.data)\{}
\NormalTok{  ns <-}\StringTok{ }\NormalTok{session}\OperatorTok{$}\NormalTok{ns}
 
\NormalTok{  output}\OperatorTok{$}\NormalTok{randomized <-}\StringTok{ }\KeywordTok{renderValueBox}\NormalTok{(\{}
    \KeywordTok{valueBox}\NormalTok{(}\DataTypeTok{value =} \KeywordTok{nrow}\NormalTok{(}\KeywordTok{baseline.data}\NormalTok{()), }\DataTypeTok{subtitle =} \StringTok{"Randomized"}\NormalTok{, }\DataTypeTok{color =} \StringTok{"green"}\NormalTok{)}
\NormalTok{  \})}
  
\NormalTok{  output}\OperatorTok{$}\NormalTok{withdrew <-}\StringTok{ }\KeywordTok{renderValueBox}\NormalTok{(\{}
    
    \CommentTok{## Fill in the value of withdrawn}
\NormalTok{    no <-}\StringTok{ }\KeywordTok{baseline.data}\NormalTok{() }\OperatorTok\StringTok{ }\KeywordTok{filter}\NormalTok{(withdrawn }\OperatorTok{==}\StringTok{ }\OtherTok{TRUE}\NormalTok{) }\OperatorTok\StringTok{ }\KeywordTok{nrow}\NormalTok{()}
    \KeywordTok{valueBox}\NormalTok{(}\DataTypeTok{value =}\NormalTok{ no, }\DataTypeTok{subtitle =} \StringTok{"Withdrawn"}\NormalTok{, }\DataTypeTok{color =} \StringTok{"red"}\NormalTok{)}
\NormalTok{  \})}
  
\NormalTok{  output}\OperatorTok{$}\NormalTok{active <-}\StringTok{ }\KeywordTok{renderValueBox}\NormalTok{(\{}
\NormalTok{    no <-}\StringTok{ }\KeywordTok{baseline.data}\NormalTok{() }\OperatorTok\StringTok{ }\KeywordTok{filter}\NormalTok{(withdrawn }\OperatorTok{==}\StringTok{ }\OtherTok{FALSE}\NormalTok{) }\OperatorTok\StringTok{ }\KeywordTok{nrow}\NormalTok{()}
    \KeywordTok{valueBox}\NormalTok{(}\DataTypeTok{value =}\NormalTok{ no, }\DataTypeTok{subtitle =} \StringTok{"Currently active"}\NormalTok{, }\DataTypeTok{color =} \StringTok{"blue"}\NormalTok{)}
\NormalTok{  \})}
  
\NormalTok{  output}\OperatorTok{$}\NormalTok{recruitplot <-}\StringTok{ }\KeywordTok{renderPlotly}\NormalTok{(\{}
    
    \CommentTok{## Enrolment plot data }
    \KeywordTok{plot_recruitment}\NormalTok{(}\KeywordTok{baseline.data}\NormalTok{(), }\StringTok{"date_random.date"}\NormalTok{, }\StringTok{"centre"}\NormalTok{)}
    
    
\NormalTok{  \})}
\NormalTok{\}}
\end{Highlighting}
\end{Shaded}

\hypertarget{mod_recruit.r}{%
\subsubsection{mod\_recruit.R}\label{mod_recruit.r}}

\hypertarget{mod_consent.r}{%
\subsubsection{mod\_consent.R}\label{mod_consent.r}}

\hypertarget{plot_recruitment.r}{%
\subsubsection{plot\_recruitment.R}\label{plot_recruitment.r}}

\hypertarget{run_app.r}{%
\subsubsection{run\_app.R}\label{run_app.r}}


\end{document}
